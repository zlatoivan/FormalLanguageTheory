\documentclass[12pt]{article}

\usepackage[T2A]{fontenc}  % кодировка шрифта
\usepackage[utf8]{inputenc}  % кодировка русских букв
\usepackage[english, russian]{babel}  % пакет для работы с языками

\usepackage[letterpaper,top=2cm,bottom=2cm,left=3cm,right=3cm,marginparwidth=1.75cm]{geometry} % формат страницы
\usepackage{enumitem}  % для компактных списков
\usepackage{listings}  % чтоб выводить строки 1 в 1


\title{Лабораторная работа №5\\"РБНФ в грамматику" (Вариант 5)\\по курсу Теория Формальных Языков}
\author{Златовратский Иван}
\date{Март 2023}

\begin{document}
    \maketitle

	\tableofcontents
	\clearpage


    \section{Условие лабораторной работы}


    \subsection{Базовое условие}
    \begin{enumerate}
        \item Необходимо предложить грамматику описания РБНФ.
        \item Параметризованными токенами грамматики могут выступать, например:
        \begin{itemize}
            \item символы начала и конца нетерминала, а также начала и конца итерации, опционального вхождения и альтернативы;
            \item символы разделения левой и правой части правила и правил друг от друга;
            \item способ объявления стартового нетерминала и обозначение для $\varepsilon$.
        \end{itemize}
        \item Из входа читается описание РБНФ в заданном синтаксисе.
        \item Требуется построить классическую КС-грамматику, ему эквивалентную.
        \item Имена для новых нетерминалов должны генерироваться так, чтобы был понятен их смысл, и откуда они берутся.
    \end{enumerate}

    \subsection{Дополнительные задания}
    \begin{itemize}
        \item (+2 балла) Язык нетерминальных и терминальных символов (в форме регулярных выражений)
        \item (+1–3 балла) Описание грамматики, документацию и пример работы программы оформить в latex. Число баллов зависит от качества оформления.
        \item (+3 балла) Параметры синтаксиса выходной грамматики также читать из (другого) файла.
        \item (+3 балла) Язык Lua
    \end{itemize}

    \clearpage


    \section{Входные данные}

    Входные данные задаются в четырех файлах:
    \begin{itemize}[noitemsep,topsep=0pt,parsep=0pt,partopsep=0pt]
        \item syntaxRBNF.txt
        \item syntaxCF.txt
        \item regex.txt
        \item RBNF.txt
    \end{itemize}


    \subsection{Синтаксис входной РБНФ грамматики}

    Синтаксис входной РБНФ грамматики задается в файле "syntaxRBNF.txt".

    \subsubsection*{Формат файла}
    \begin{itemize}
        \item В файле уже заданы названия токенов и их значения по умолчанию.
        \item Пользователю разрешено изменять только значения токенов, которые задаются в кавычках и количество пробелов вокруг знака равенства, а также, порядок токенов.
    \end{itemize}

    \subsubsection*{Значения токенов}
    \begin{table}[hbt!]
        \begin{tabular}{|l|l|}
            \hline
            Токен & Опиание \\
            \hline
            nonTerminalStart = "\_" & токен начала нетерминала \\
            nonTerminalEnd = "\_" & токен конца нетерминала \\
            arrow = ":=" & токен, стоящий между именем нетерминала и его значением \\
            iterStart = "\{" & токен начала итерации \\
            iterEnd = "\}" & токен конца итерации \\
            optionalStart = "[" & начала опционального вхождения \\
            optionalEnd = "]" & токен конца опционального вхождения \\
            necessaryStart = "(" & токен начала обязательного вхождения \\
            necessaryEnd = ")" & токен конца обязательного вхождения \\
            alternative = "$|$" & токен, разделяющий альтернативные правила \\
            epsilon = "eps" & токен пустого слова \\
            \hline
        \end{tabular}
        \caption{Файл "syntaxRBNF.txt"}
    \end{table}

    \subsubsection*{Формат токенов}
    \begin{itemize}
        \item Количество символов в токене должно быть от одного или больше.
        \item Значения токенов не должны повторяться (могут повторяться только nonTerminalStart и nonTerminalEnd)
        \item Из ASCII с 33 по 126 символы:
        \begin{itemize}
            \item Разрешены:   \{ \} [ ] ( ) / | \ ! ? " ' ` , . : ; \_ ~ < = > \& \# @ 0-9 A-Z a-z
            \item Запрещены:  \% \$ \^{} $\ast$ + -
        \end{itemize}
    \end{itemize}


    \subsection{Синтаксис выходной КС грамматики}

    Синтаксис выходной грамматики задается в файле "syntaxCF.txt".

    \subsubsection*{Формат файла}
    \begin{itemize}
        \item В файле уже заданы названия токенов и их значения по умолчанию.
        \item Пользователю разрешено изменять только значения токенов, которые задаются в кавычках и количество пробелов вокруг знака равенства, а также, порядок токенов.
    \end{itemize}

    \subsubsection*{Значения токенов}
    \begin{table}[hbt!]
        \begin{tabular}{|l|l|}
            \hline
            Токен & Опиание \\
            \hline
            nonTerminalStart = "[" & токен начала нетерминала \\
            nonTerminalEnd = "]" & токен конца нетерминала \\
            arrow = "->" & токен, стоящий между именем нетерминала и его значением \\
            alternative = "$|$" & токен, разделяющий альтернативные правила \\
            epsilon = "\$" & токен пустого слова \\
            \hline
        \end{tabular}
        \caption{Файл "syntaxCF.txt"}
    \end{table}

    \subsubsection*{Формат токенов}
    \begin{itemize}
        \item Количество символов в токене должно быть от одного или больше.
        \item Значения токенов могут повторяться.
        \item Из ASCII с 33 по 126 символы:
        \begin{itemize}
            \item Разрешены:   \{ \} [ ] ( ) / | \ ! ? " ' ` , . : ; \_ ~ < = > \& \# @ 0-9 A-Z a-z\\ \$ \^{} $\ast$ + -
            \item Запрещены:  \%
        \end{itemize}
    \end{itemize}

    \clearpage


    \subsection{Ввод регулярного выражения}

    Регулярное выражение задается в файле "syntaxCF.txt".

    \subsubsection*{Формат файла}
    \begin{itemize}
        \item В файле уже заданы названия токенов и их значения по умолчанию.
        \item Пользователю разрешено изменять только значения токенов, которые задаются в кавычках и количество пробелов вокруг знака равенства, а также
        \item Порядок токенов изменять нельзя.
    \end{itemize}

    \subsubsection*{Значения токенов}
    \begin{table}[hbt!]
        \centering
        \begin{tabular}{|l|l|}
            \hline
            Токен & Опиание \\
            \hline
            Nonterminal ::= [A-Z|0|1][A-z]* & Регуляргое выражение для нетерминалов\\
            Terminal ::= [a-z]* & Регуляргое выражение для терминалов\\
            \hline
        \end{tabular}
        \caption{Файл "regex.txt"}
    \end{table}

    \subsubsection*{Формат токенов}
    \begin{itemize}
        \item Классические регулярные выражения.
        \item Могут быть одинаковыми для нетерминалов и терминалов.
    \end{itemize}

    %\clearpage


    \hfill
    \subsection{Ввод РБНФ грамматики}

    РБНФ грамматика задается в файле "syntaxCF.txt".

    \subsubsection*{Формат файла}
    Пользователь должен ввести РБНФ грамматику в соответствии с синтаксисом и терминалы, нетерминалы в ней, удовлетворяющие регулярным выражениям.

    \subsubsection*{Пример РБНФ грамматики}
    \begin{lstlisting}
    _DightNotNull_ := 1 | 2 | 3 | 4 | 5 | 6 | 7 | 8  | 9
    _Dight_        := 0 | _DightNotNull_
    _Natural_      := (_DightNotNull_) {_Dight_}
    _Int_          := 0 | [minus] _Natural_
    \end{lstlisting}

    \clearpage


    \section{Выходные данные}

    \subsection{Возможные ошибки}

    Если входные данные неверны, то программа выведет ошибку.

    \subsubsection*{Варианты ошибок}
    \begin{itemize}
        \item 'Error (syntax): Not right arrow: \_A\_ :=== \_A\_b'
        \item 'Error (syntax): ] <- not found in : (\_DightNotNull\_)[\_Dight\_'
        \item 'Error (syntax): Not equal number of \{ and \}   in: (\_DightNotNull\_)\{\_Dight\_\}\}'
        \item 'Error (syntax): No expression in brackets'
        \item 'Error (regex): '\#a' <- Left nonterm doesn't match to it's regex'
        \item 'Error (regex): '\#b' <- Right nonterm doesn't match to it's regex'
        \item ‘Error (regex): 'A' <- Term doesn't match to it's regex’
    \end{itemize}

    \subsection{Вывод при верных данных}

    Если входные данные верны, программа выведет этапы работы парсера, где можно увидеть процесс замены вхождений на новые нетерминалы, и после этого - КС грамматику.

    \clearpage


    \subsubsection*{\large Простой пример}
    Синтаксис РБНФ и КС грамматик, а также, регулярные выражения заданы по умолчанию.

    \hfill \\
    {\bfseries RBNF.txt:}
    \begin{lstlisting}
        _R_ := {_S_}[[_S_](_S_)]
    \end{lstlisting}

    \hfill \\  % - перенос строки
    \noindent{\bfseries Результет работы программы:}
    \begin{lstlisting}
    Parser:
        _R_ := {_S_}[[_S_](_S_)]

            {_S_}[[_S_](_S_)]     change: {_S_} -> _Nt1_
            _Nt1_[[_S_](_S_)]

            _Nt1_[[_S_](_S_)]     change: [_S_] -> _Nt2_
            _Nt1_[_Nt2_(_S_)]

            _Nt1_[_Nt2_(_S_)]     change: (_S_) -> _S_
            _Nt1_[_Nt2__S_]

            _Nt1_[_Nt2__S_]     change: [_Nt2__S_] -> _Nt3_
            _Nt1__Nt3_


    CF grammar:
        [Nt1] -> [S][Nt1] | $
        [Nt2] -> [S] | $
        [Nt3] -> [Nt2][S] | $
        [R] -> [Nt1][Nt3]
    \end{lstlisting}

    \clearpage


    \subsubsection*{\large Пример "Integer"}
    Синтаксис РБНФ и КС грамматик, а также, регулярные выражения заданы по умолчанию.

    \hfill \\
    {\bfseries RBNF.txt:}
    \begin{lstlisting}
        _DightNot0_ := 1 | 2 | 3 | 4 | 5 | 6 | 7 | 8  | 9
        _Dight_        := 0 | _DightNot0_
        _Natural_      := (_DightNot0_) {_Dight_}
        _Int_          := 0 | [minus] _Natural_
    \end{lstlisting}

    \hfill \\
    \noindent{\bfseries Результет работы программы:}
    \begin{lstlisting}
    Parser:
        _DightNot0_ := 1|2|3|4|5|6|7|8|9

        _Dight_ := 0|_DightNot0_

        _Natural_ := (_DightNot0_){_Dight_}

            (_DightNot0_){_Dight_}    change: {_Dight_} -> _Nt1_
            (_DightNot0_)_Nt1_

            (_DightNot0_)_Nt1_   change: (_DightNot0_) -> _DightNot0_
            _DightNot0__Nt1_

        _Int_ := 0|[minus]_Natural_

            0|[minus]_Natural_    change: [minus] -> _Nt2_
            0|_Nt2__Natural_

    CF grammar:
        [DightNot0] -> 1 | 2 | 3 | 4 | 5 | 6 | 7 | 8 | 9
        [Dight] -> 0 | [DightNot0]
        [Nt1] -> [Dight][Nt1] | $
        [Natural] -> [DightNot0][Nt1]
        [Nt2] -> minus | $
        [Int] -> 0 | [Nt2][Natural]
    \end{lstlisting}

    \clearpage


    \subsubsection*{\large Пример грамматики с очень нестандартным синтаксисом}

    \hfill \\
    {\bfseries syntaxRBNF.txt:}
    \begin{lstlisting}
        nonTerminalStart = "<"
        nonTerminalEnd = ">"
        arrow = ":===="
        iterStart = "IS"
        iterEnd = "IE"
        optionalStart = "OS"
        optionalEnd = "OE"
        necessaryStart = "NS"
        necessaryEnd = "NE"
        alternative = "|||"
        epsilon = "eps"
    \end{lstlisting}

    \hfill \\
    {\bfseries syntaxCF.txt:}
    \begin{lstlisting}
        nonTerminalStart = "-+"
        nonTerminalEnd = "+-"
        arrow = "->^"
        alternative = "*"
        epsilon = "$"
    \end{lstlisting}

    \hfill \\
    {\bfseries regex.txt:}
    \begin{lstlisting}
        Nonterminal ::= [A-Z|0|1][A-z]*
        Terminal ::= [a-z]*
    \end{lstlisting}

    \hfill \\
    {\bfseries RBNF.txt:}
    \begin{lstlisting}
        <R> :==== OS<S>OEIS<S>IENS<S>NE ||| eps
        <Q> :==== NS<X>NENSaNEb
    \end{lstlisting}

    \hfill \\
    \noindent{\bfseries Результет работы программы:}
    \begin{lstlisting}
    Parser:
        <R> := OS<S>OEIS<S>IENS<S>NE|||eps

            OS<S>OEIS<S>IENS<S>NE|||eps     change: IS<S>IE -> <Nt1>
            OS<S>OE<Nt1>NS<S>NE|||eps

            OS<S>OE<Nt1>NS<S>NE|||eps     change: OS<S>OE -> <Nt2>
            <Nt2><Nt1>NS<S>NE|||eps

            <Nt2><Nt1>NS<S>NE|||eps     change: NS<S>NE -> <S>
            <Nt2><Nt1><S>|||eps

        <Q> := NS<X>NENSaNEb

            NS<X>NENSaNEb     change: NS<X>NE -> <X>
            <X>NSaNEb

            <X>NSaNEb     change: NSaNE -> a
            <X>ab


    CF grammar:
        -+Nt1+- ->^ -+S+--+Nt1+- * $
        -+Nt2+- ->^ -+S+- * $
        -+R+- ->^ -+Nt2+--+Nt1+--+S+- * $
        -+Q+- ->^ -+X+-ab

    \end{lstlisting}

    \clearpage


    \section{Запуск программы}

    Все входные данные находятся в папке "/input":
    \begin{lstlisting}
        /input/syntaxRBNF.txt
        /input/syntaxCF.txt
        /input/regex.txt
        /input/RBNF.txt
    \end{lstlisting}

    \hfill \\
    Программа находится в файле "/lua/main.lua"\\
    \hfill \\
    Запуск прграммы осуществяется окмандой:
    \begin{lstlisting}
        lua54 main.lua
    \end{lstlisting}

\end{document}